######################################################################################
#
#    #####   #     #   ####   #     #
#    #       ##   ##   #      ##   ##
#    #       #  #  #   ##     #  #  #
#    #       #     #   #      #     #
#    #####   #     #   ####   #     #
#
#  Copyright ECMWF
#  CMEM gribapi I/O
#
#########################################################################################
# Main programm is cmem_main.F90
# input files: forcing*
# input config: input (configuration of namelists)
# output files: out*
#########################################################################################
#
# FILE DIMENSION IDENTICAL FOR EACH INPUT FILE: N = LATITUDE x LONGITUDE
#
#
##########################################################################################
#Prepare the forcing file according to your run config. Here grib input files are required
# Each name of input file starts with 'forcing' 
# forcing files must be consistent in terms of grid size (checked in rdcmemgribapiinfo.F90)
# example of foring file is provided (global file) 
#  FORCING and OUTPUT FILES IN GRIB    --->   CFINOUT='gribapi' 
#  Default is gripapi since January 2009
#
#  GRIBAPI programs developed for encoding and decoding WMO FM-92 GRIB edition 1 and edition 2 messages
#  For CMEM they requiere the same I/O. GribAPI however require one additional file used as template for the outputs 
#  (copy of SWLV1.grib in SWLV1.tmpl).
#  
#  To use gribapi, which is the new default of CMEM, you need to install GRIBAPI
# http://www.ecmwf.int/products/data/software/download/grib_api.html 
# in order to indicate the PATH for the template file, do:
# setenv GRIB_TEMPLATES_PATH /YourPathWhereTemplateFileIs/ 
#
############################################
#1) Meteo
#9 files required for any option:
############################################

SD.grib  snow depth in water equiv (m)
RSN.grib snow density (kg/m3)

STL1.grib Soil temperature level 1 (K)
STL2.grib Soil temperature level 2 (K)
STL3.grib Soil temperature level 3 (K)
TSKIN.grib Skin temperature (K)

SWVL1.grib volumetric soil moist level1 (m3/m3)
SWVL2.grib volumetric soil moist level2 (m3/m3) 
SWVL3.grib volumetric soil moist level3 (m3/m3)  

SWVL1.tmpl copy of SWVL1.grib to be used as template for the output  (only required if gribapi is
used)

###############################################################
# 2) soil conditions
#3 file Required for any option 
################################################################

 - Z.grib (geopotential at surface in m2/s2)
 - sand.grib fraction of sandy textured soil (%) range 0-100
 - clay.grib fraction of clay textured soil (%) range 0-100

##########################################################
#3) Vegetation
#5 files required for any option
##########################################################
# in case ecoclimap is choosen, the following files are required
 
 - ECOCVL.grib  fraction of low veg (-) 
 - ECOCVH.grib  fraction of high veg (-)
#These fractions are such as their sum is the complement to one of the bare soil fraction:
#bare soil frac = 1- cvh+cvl

 - ECOTVL.grib  low veg type 
 - ECOTVH.grib  high veg type
 - ECOWAT.grib water fraction

# or when Tessel option is choosen:

 - CVL.grib fraction of low veg (-)
 - CVH.grib fraction of high veg (-)
#Appart for desert areas these fractions are such as their sum is 1
#Accordingly a weigting function is applied to these fractions to compute bare soil

 - TVL.grib low veg type
 - TVH.grib high veg type
 - LSM.grib land fraction (water fraction is set = 1-LSM) (-)



#### useful information:
# ECOCLIMAP types are:
#No vegetation: 0
#High vegetation: 1 Decidious forests; 2 Coniferous forests; 3 Rain forests
#Low vegetation: 4 C3 Grasslands; 5 C4 Grasslands; 6 C3 Crops; 7 C4 Crops
#
# TESSEL types are:
#High vegetation: 3 Evergreen Needleleaf Trees; 4 Deciduous Needleleaf Trees; 5 eciduous Broadleaf Trees;
#                 6  Evergreen Broadleaf Trees; 18 Mixed Forest/woodland; 19 Interrupted Forest
#Low vegetation:  1 Crops, Mixed Farming; 2 Short Grass; 7 Tall Grass; 9 Tundra; 10 Irrigated Crops;
#                 11 Semidesert; 13 Bogs and Marshes; 16 Evergreen Shrubs; 
#                 17 Deciduous Shrubs; 20 Water and Land Mixtures

# Vegetation tables are provided in vegtable.F90

#########################################
# 4)  VEGETATION LAI
# 1 file if Ecoclimap

  -ECOLAIL.grib  lai of low veg for each pixel

# 0 file if tessel
###########################################
# For Otpion CITVEG = 'Tair' or Otpion CIATM='Pellarin' or CIATM='Ulaby' (in input) 

 - 2T.grib (air temp at 2m - K)



###############################################
# Vertical discretization in CMEM

# CMEM considers two vertical grids:

 #1- Land Surface Model vertical resolution for soil moisture
 # and soil temperature profiles (read by CMEM in rdcmem*) 
 # Default soil layers depths are given in cmem_setup.F90
 # NLAY_SOIL_LS=3 and soil depths z_lsm(:) defined
 # the deepest layer define the depth of the soil column
 # If you want to enter an other input vertical grid:
 # 	* define in the input file NLAY_SOIL_LS
 #  * add a file called LSM_VERTICAL_REOL.asc with depth of each layer
 # (see example for wilheit in io_sample)

 #2- Microwave fine vertical grid: 
 # a: With CISMR='Fresnel' (default) the number of layer is one (NLAY_SOIL_MW=1)
 # b: CISMR='Wilheit' must be selected in the file 'input' to allow multi-layer MW emission modelling 
 # by default it uses NLAY_SOIL_MW=10 layers in cmem_setup.F90
 # to change it, give a value to NLAY_SOIL_MW in the input file
 # the layer thicknesses are computed automatically in cmem_soil.F90
 # the input sm and st profiles read in rdcmem* are interpolated to the 
 # NLAY_SOIL_MW layers and used in wilheit



